\section{PHP}

\begin{itemize}
	\item \textbf{index.php}: è la pagina iniziale del sito che ne mostra le funzionalità. E' una pagina dinamica in quanto le voci del menu cambiano se l'utente è loggato o meno;
	\item \textbf{login.php}: è la pagina per accedere al sito con il proprio profilo attraverso il nome del gruppo e la propria password. Nel caso il login fallisca viene mostrato un messaggio d'errore, mentre se si accede correttamente si viene reindirizzati in user.php;
	\item \textbf{logout.php}: ha il compito di distruggere la sessione;
	\item \textbf{registration.php}: è la pagina per creare un proprio profilo all'interno del sito. Nel form SCRIVERE DEI CONTROLLI
	\item \textbf{contacts.php}: mostra le pagine social e la mail per rimanere in contatto e informati con SWEgo. E' una pagina dinamica per lo stesso motivo di index.php;
	\item \textbf{tracking.php}: è la pagina utilizzata per inserire un nuovo tracciamento tra Use Case e requisiti;
	\item \textbf{user.php}: è la pagina del profilo dell'utente dove può visualizzare tutti suoi dati; in particolare l'utente loggato può creare dei grafici a torta, attraverso l'apposito link, che mostrano il riassunto dei requisiti obbligatori, desiderabili e facoltativi soddisfatti in quel determinato momento. Da questa pagina è possibile anche creare le immagini degli Use Case attraverso PlantUml e fare il logout dal proprio profilo;
	\item \textbf{viewRequirement.php, viewTracking.php, viewUsecase.php}: sono le pagine che rispettivamente recuperano e mostrano i dati relativi ai requisiti, Use Case e tracciamento Use Case-requisiti del progetto dell'utente;
	\item \textbf{insertActor.php, insertSource.php}: sono le pagine che recuperano e mostrano le informazioni degli attori e delle fonti del progetto dell'utente; danno anche la possibilità di inserirne di nuovi attraverso l'apposito form;
	\item \textbf{insertRequirement.php, insertUseCase}: sono le pagine che danno la possibilità di inserire nuovi requisiti e Use Case nel progetto;
	\item \textbf{deleteX.php}: sono le pagine che danno la possibilità di eliminare dal progetto, e quindi dal database, attori, fonti, requisiti, Use Case e il tracciamento Use Case/requisiti creati precedentemente dall'utente; 
	\item \textbf{modifyX.php}: sono le pagine che danno la possibilità di modificare i dati di Use Case, requisiti, attori, fonti e del tracciamento precedentemente inseriti.
\end{itemize}

AGGIUNGEREI I CONTROLLI FATTI SUI FORM MA NON LI SO SCRIVERE BENE