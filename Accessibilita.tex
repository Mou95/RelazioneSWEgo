\section{Accessibilità}
\subsection{Separazione tra struttura, presentazione e comportamento}
Per migliorare l'accesso al sito a qualsiasi categoria di utenti è stata mantenuta la separazione tra struttura, presentazione e comportamento. \\
La prima è stata sviluppata tramite documenti HTML5. Questi richiamano i fogli di stile esterni CSS, che implementano la presentazione, e gli script esterni di JavaScript che ne determinano il comportamento.
\subsection{Tag meta}
Per ogni pagina web sono stati inseriti i seguenti tag meta:
\begin{itemize}
	\item \textbf{title}: titolo;
	\item \textbf{description}: descrizione sintetica;
	\item \textbf{keywords}: lista di parole chiave che permette di specificare gli argomenti trattati;
	\item \textbf{author}: indica gli autori della pagina;
	\item \textbf{viewport}: utilizzato per "comunicare" al browser come adattare il sito per dispositivi con diverse dimensioni.
\end{itemize}


\subsection{Percepibilità}
La percepibilità consiste in: 
	\begin{itemize}
		\item fornire alternative testuali per qualsiasi contenuto non testuale, in modo che questo possa essere trasformato in altre forme fruibili secondo le necessità degli utenti;
		\item rendere più semplice agli utenti la visione dei contenuti.
	\end{itemize}
	In particolare il gruppo ha seguito queste regole:
	\begin{itemize}
		\item attributo \textit{alt} con un appropriato testo sostitutivo a tutte le immagini;
		\item non sono state utilizzate immagini per riportare il testo, rendendo il contenuto informativo accessibile anche agli utenti che utilizzano gli screen reader;
		\item attributo \textit{scope= "col"} per specificare la cella header di una colonna, o gruppo di colonne in una tabella;
		\item un tag \textit{label} che descrive la finalità di ogni textfield di ogni form;
		\item un tag \textit{legend} per tutti i fieldset, utile a descriverne il contenuto.
	\end{itemize}
\subsubsection{Colori}
Per rendere il più semplice possibile agli utenti la visione dei contenuti è stato adottato un appropriato schema di colori, il quale garantisce un certo contrasto cromatico.
	In particolare, miriamo a dare meno disagi possibili agli utenti affetti da particolari patologie alla vista, quali deuteranopia, protanopia e tritanopia. \\
	Gli screenshot qui sotto sono una simulazione di come un utente appartenente a questa categoria vede il nostro sito.\\
	\begin{figure}
	\centering
		\includegraphics[scale=0.4]{img/normale.png}\\[1cm] \caption{Pagina di login vista da un utente normale.}
	\end{figure}
	\begin{figure}
	\centering
		\includegraphics[scale=0.3]{img/deuteranopia.png}\\[1cm] \caption{Pagina di login vista da un utente affetto da deuteranopia}
	\end{figure}
	\begin{figure}
	\centering
		\includegraphics[scale=0.3]{img/protanopia.png}\\[1cm] \caption{Pagina di login vista da un utente affetto da protanopia}
	\end{figure}
	\begin{figure}
	\centering
		\includegraphics[scale=0.3]{img/tritanopia.png}\\[1cm] \caption{Pagina di login vista da un utente affetto da tritanopia}
	\end{figure}
		\begin{figure}
	\centering
		\includegraphics[scale=0.8]{img/normale_mobile.jpeg}\\[1cm] \caption{Pagina mobile di login vista da un utente normale.}
	\end{figure}
	\begin{figure}
	\centering
		\includegraphics[scale=0.8]{img/deuteranopia_mobile.png}\\[1cm] \caption{Pagina mobile di login vista da un utente affetto da deuteranopia}
	\end{figure}
	\begin{figure}
	\centering
		\includegraphics[scale=0.8]{img/protanopia_mobile.png}\\[1cm] \caption{Pagina mobile di login vista da un utente affetto da protanopia}
	\end{figure}
	\begin{figure}
	\centering
		\includegraphics[scale=0.8]{img/tritanopia_mobile.png}\\[1cm] \caption{Pagina mobile di login vista da un utente affetto da tritanopia}
	\end{figure}
	\newpage
\subsection{Usabilità}
Al fine di migliorare l'esperienza d'uso degli utenti sono state inserite le seguenti facilitazioni:
\begin{itemize}
	\item \textbf{tabindex}: non è stato ridefinito il comportamento del pulsante tab tramite l'attributo tabindex, in quanto il gruppo ritiene quello di default agevole per la navigazione;
	\item \textbf{link per spostarsi al contenuto}: prima della barra di navigazione è stato inserito un link nascosto per saltarla, permettendo agli utenti che utilizzano uno screen reader di passare direttamente al contenuto;
\end{itemize}
Inoltre per rendere ben navigabile il sito si è fatto uso di:
\begin{itemize}
	\item \textbf{breadcrumb} che permette di capire in che pagina ci si trova;
	\item \textbf{mappa del sito}.
\end{itemize}

\subsection{Comprensibilità}
Le informazioni del sito devono risultare comprensibili per gli utenti. A tal scopo si è fatto utilizzo di:
	\begin{itemize}
		\item \textit{lang=en} per specificare che alcune parole sono inglesi;
		\item menù simili per tutte le pagine, rendendo la voce di menù della pagina attuale non cliccabile;
		\item apposite label e istruzioni che aiutano l'utente a inserire dati corretti;
		\item un gruppo di \textit{checkbox} al posto di \textit{select multiple}, in quanto risulta molto più comprensibile e di più facile utilizzo.
	\end{itemize}
\subsection{Robustezza}
Il gruppo si è prefisso come scopo la più alta compatibilità possibile tra browser, rendendo il sito utilizzabile fino a Internet Explorer 8. Questo è stato reso possibile facendo utilizzo di regole CSS e funzioni JavaScript compatibili
con quest'ultimo. \\
Inoltre il sito è stato testato in diversi altri browser (Google Chrome 59.0, Mozilla Firefox 53.0 e Microsoft Edge), dove l'aspetto e le funzionalità rimangono pressoché inalterate.

\subsection{Screen Reader}
Per testare la compatibilità del sito con uno Screen Reader, è stato scelto di usare il programma gratuito NVDA. Alcuni membri del gruppo hanno utilizzato tale programma per navigare all’interno del sito, sia nell’area utente sia nell’area di amministrazione, senza riscontrare particolari problemi. Il sito risulta quindi utilizzabile anche da parte di una persona che non può usare un browser canonico, ma deve fare affidamento ad uno screen reader.