\section{Test}
\subsection{Validazione}
Per la validazione, è stato usato il validatore online messo a disposizione da W3C. Per validare ogni pagina, si sono seguiti i seguenti passi:
\begin{itemize}
	\item esecuzione del file nel browser;
	\item copia del codice sorgente;
	\item controllo con il validatore tramite "validate by input".
\end{itemize}
Tutte le pagine e i file CSS hanno superato questo test.
\subsection{Browser}
Senza riscontrare grosse differenze, il sito è stato testato sui seguenti browser:
\begin{itemize}
	\item Mozilla Firefox 53.0;
	\item Google Chrome 59.0;
	\item Internet Explorer 8, 9 , 10, 11;
	\item Microsoft Edge.
\end{itemize}
\subsubsection{JavaScript}
Nel caso in cui JavaScript risulti disattivato non viene rilevato alcun tipo di errore, e il sito può garantire la maggior parte delle sue funzionalità. In particolare resta possibile visualizzare, inserire, modificare ed eliminare tutti i dati del progetto, ma non ricercare requisiti o Use Case tramite l'apposito input di ricerca.
