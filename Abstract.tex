\section{Abstract}
Il sito è stato realizzato con l'intenzione di fornire uno strumento di supporto per il tracciamento e la gestione di Use Case e requisiti, aspetto fondamentale nella realizzazione di un prodotto software.
Nel nostro percorso universitario abbiamo affrontato il corso di Ingegneria del Software, che prevede lo svolgimento di un progetto di gruppo, della durata di un semestre. Questo progetto, dedicato allo sviluppo di un prodotto software proposto da   aziende esterne all'università di Padova, ha cercato di insegnarci come affrontare nel miglior modo possibile un problema di discrete dimensioni, per farci maturare sia da un punto di vista delle nostre conoscenze, sia nel nostro lato umano. \\
Proprio nel primo periodo, il nostro gruppo si è reso conto della mancanza di uno strumento da utilizzare per raccogliere e poter gestire i requisiti del proponente individuati durante l'analisi del problema, e i relativi casi d'uso. \\
Il tracciamento dei requisiti è cruciale per poter controllare se tutte le richieste del cliente sono state prese in considerazione e per poter valutare, in qualsiasi momento del progetto, quali di questi requisiti vengono soddisfatti dal prodotto, e quali no. Per questo motivo il gruppo SWEgo ha deciso di creare questo sito per offrire una serie di funzionalità di fondamentale importanza nella realizzazione di un prodotto Software. \\
Ai nostri utenti diamo la possibilità di creare tutti i propri requisiti, decidendo per ognuno:
\begin{itemize}
	\item \textbf{codice};
	\item \textbf{nome};
	\item \textbf{descrizione};
	\item \textbf{tipo} (funzionale, di vincolo, di qualita o ???);
	\item \textbf{importanza} (obbligatorio, desiderabile o facoltativo da implementare);
	\item \textbf{stato di soddisfacimento};
	\item \textbf{fonte};
\end{itemize}
In relazione ai requisiti possono essere tracciati anche gli use case, cioè come valutare ogni requisito focalizzandosi sugli attori che interagiscono col sistema. Per gli use case può essere specificato:
\begin{itemize}
	\item \textbf{codice};
	\item \textbf{nome};
	\item \textbf{descrizione};
	\item \textbf{scenario principale};
	\item \textbf{scenario alternativo};
	\item \textbf{padre};
	\item \textbf{estensioni};
	\item \textbf{inclusioni};
	\item \textbf{i requisti associati};
	\item \textbf{l'attore coinvolto};
\end{itemize}

