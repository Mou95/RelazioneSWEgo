\section{Comportamento}
La parte relativa al comportamento delle pagine è stata gestita con JavaScript. In particolare si è fatto uso dei seguenti script:
\begin{itemize}
	\item \textbf{features.js}: ha il compito di far comparire o nascondere la descrizione dei vari div presenti nella homepage del sito, cliccando l'apposito link;
	\item \textbf{nav.js}: ha il compito di far comparire o nascondere il menù del sito nella versione mobile, cliccando l'apposito pulsante;
	\item \textbf{requirement.js}: ha il compito di filtrare la tabella dei requisiti, in base a quanto l'utente scrive nell'apposito input per la ricerca. I requisiti vengono selezionati solo se il codice o il nome corrispondono all'input dell'utente. Se la ricerca non da risultati, viene mostrata una descrizione appropriata;
	\item \textbf{useCase.js}: stesso compito di "requirement.js" per la tabella degli Use Case (il filtro qui viene applicato su codice e nome).
	\item \textbf{user.js}: ha il compito di costruire i grafici e far comparire o nascondere il form del cambio password e i grafici stessi nella pagine del profilo utente.
\end{itemize}