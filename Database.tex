\section{Gestione dei dati}
Per la gestione e memorizzazione dei dati è stato usato un database MySQL. La struttura delle tabelle è stata pensata per evitare la presenza di dati ridondanti, o comportamenti anomali durante l’aggiornamento dei dati, fornendo così una struttura stabile e adatta alle modifiche. \\
Le principali tabelle utilizzate sono:
\begin{itemize}
	\item \textbf{actors}: contiene i dati degli attori inseriti dai vari utenti;
	\item \textbf{requirements}: contiene i dati dei requisiti inseriti dagli utenti;
	\item \textbf{sources}: contiene i dati delle fonti inserite dagli utenti;
	\item \textbf{usecase}: contiene i dati degli Use Case inseriti dagli utenti;
	\item \textbf{users}: contiene i dati di registrazione di ogni utente;
	\item \textbf{usecaseactors}: contiene gli attori coinvolti in ognuno degli Use Case;
	\item \textbf{usecaseextensions}: contiene le estensioni, se presenti, di ognuno degli Use Case;
	\item \textbf{usecaseinclusions}: contiene le inclusioni, se presenti, di ognuno degli Use Case;
	\item \textbf{usecaserequirements}: contiene il tracciamento tra Use Case e requisiti;
\end{itemize}  