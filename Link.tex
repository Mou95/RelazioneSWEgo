\section{Link}
Nella realizzazione del sito sono state tenute in considerazione le consuetudini per i link, ossia tenere il testo sottolineato per indicare la presenza di un link e cambiarne il colore una volta visitato. In alcuni casi è stato deciso di non rispettare tali pratiche, in particolare:
\begin{itemize}
	\item nei link del menù, poiché gli utenti sono ormai abituati ad utilizzare menù con link visitati e non visitati indistinti. Inoltre, è molto probabile che i link nel menù principale diventino tutti visitati dopo un beve periodo di utilizzo del sito, rendendo quindi superflua la distinzione;
	\item nei link del breadcrumb, per lo stesso motivo dei link del menù;
	\item nelle pagine index e user dove sono presenti dei link per nascondere o mostrare del contenuto della pagina;
	\item nel footer il colore del link per la mappa del sito è grigio per non essere confuso con il colore dello sfondo.
\end{itemize}
