\section{Presentazione}
Per la rappresentazione dell'interfaccia grafica del sito si è rispettato lo standard CSS3.\\
L'uso di specifiche proprietà del CSS3 sono state limitate al minimo in modo da essere il più possibile compatibile con versioni di browser meno recenti e permettere un degrado elegante anche in caso di mal funzionamento.
Per garantire una buona usabilità del sito i colori per la realizzazione del sito sono stati scelti in modo da avere un buon contrasto tra testo e sfondo, in questo modo sono facilmente visibili anche da utenti affetti da particolari
difetti visivi come daltonismo.\\
Abbiamo prestato attenzione a non avere elementi lampeggianti nel sito per non creare disagi ad utenti affetti da epilessia fotosensibile.

\subsection{Divisione dei file}
Il sito non presenta problemi nel ridimensionamento della pagina grazie all'utilizzo di fogli di stile diversi a seconda della dimensione del dispositivo. In particolare nella cartella css sono presenti 3 file:
\begin{itemize} 
	\item \textbf{desktop.css}: contiene le regole di stile per dispositivi con una "width" maggiore a 520px; 
	\item \textbf{mobile.css}: contiene le regole di stile per dispositivi mobile con "width" minore a 520px;
	\item \textbf{print.css}: contiene le regole di stile relative alla stampa delle varie pagine del sito.
\end{itemize}
Le differenze tra versione desktop e mobile sono ristrette al diverso posizionamento di alcuni elementi delle pagine, mantenendo così uno stile grafico del sito omogeneo, in questo modo per l'utente abituato ad accedere alla
versione desktop non si ritrova confuso da un sito completamente diverso. \\
In particolare nella versione mobile le voci del menù possono essere visualizzate attraverso un apposito pulsante, che nella versione desktop viene nascosto, per ridurre lo spazio occupato dall'header.