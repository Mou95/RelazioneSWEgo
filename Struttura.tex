\section{Struttura}
\subsection{Struttura del sito}
Il sito è diviso in diverse aree, raggiungibili tramite il menù, a seconda se l'utente è un utente loggato, non loggato o un admin. In particolare:
\begin{itemize}
	\item \textbf{Utente non loggato}
	\begin{itemize}
		\item \textbf{Homepage}: è la homepage del sito dove si può vedere una panoramica delle funzionalità offerte;
		\item \textbf{Accedi/Registrati}: in queste pagine l'utente può accedere al sito con il proprio profilo oppure crearne uno nuovo se non si è mai registrato;
		\item \textbf{Contatti}: in questa pagina vengono mostrate le pagine social e la mail attraverso le quali è possibile contattare gli sviluppatori, o semplicemente rimanere informati sulle novità di SWEgo.
	\end{itemize}
	\item \textbf{Utente loggato}:
	\begin{itemize}
		\item \textbf{Homepage} (come sopra);
		\item \textbf{Contatti} (come sopra);
		\item \textbf{Attori}: in questa pagina si trova il riepilogo degli attori inseriti dall'utente. Un attore può essere inteso come un ruolo coperto da un certo insieme di entità interagenti col sistema (inclusi utenti umani, altri sistemi software, dispositivi hardware e così via). E' possibile aggiungere, eliminare o modificare tutti gli attori;
		\item \textbf{Use Case}: in questa pagina si trovano tutti gli Use Case inseriti dall'utente, con la possibilità di inserirne di nuovi ed eliminare o modificare quelli già presenti;
		\item \textbf{Fonti}: in questa pagina l'utente può visualizzare le fonti inserite nel sistema; una fonte può essere intesa come il soggetto dal quale un requisito è stato estrapolato;
		\item \textbf{Requisiti} (come per gli Use Case, con al posto i requisiti);
		\item \textbf{Tracciamento}: in questa pagina si può vedere un riassunto dei requisiti associati ad ogni Use Case;
		\item \textbf{Profilo}: in questa pagina l'utente può visualizzare e modificare i propri dati, visualizzare alcuni grafici riassuntivi e generare le immagini degli Use Case tramite plantIml;
	\end{itemize}
	\item \textbf{Admin} (oltre alle pagine da utente loggato):
	\begin{itemize}
		\item \textbf{Summary}: MANCA
	\end{itemize}
Da ogni pagina è raggiungibile la \textit{mappa del sito}(diversa in base alla tipologia dell'utente) attraverso il link presente nel footer.
\end{itemize}
\subsection{Struttura delle pagine}
Ogni pagina del sito è differente per i contenuti ma vi sono alcuni elementi che ricorrono in ognuna di esse:
\begin{itemize}
	\item \textbf{header}: contiene il logo del sito e i link alle pagine precedentemente descritte per ogni categoria di utente;
	\item \textbf{breadcrumb}: fa sapere all'utente in che pagina si trova in quel momento descrivendo anche il percorso seguito per arrivarci;
	\item \textbf{footer}: contiene un riferimento al gruppo, un link alla mappa del sito e le immagini che certificano che le pagine sono conformi a HTML5 e CSS3.
\end{itemize}